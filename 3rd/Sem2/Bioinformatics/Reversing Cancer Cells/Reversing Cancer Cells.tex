\documentclass[12pt,a4paper]{article}
\usepackage[utf8]{inputenc}
\usepackage[english]{babel}
\usepackage{geometry}
\usepackage{setspace}
\usepackage{times}

\geometry{margin=2.5cm}
\onehalfspacing

\title{The discovery of a molecular switch that reverses cancer cells}
\author{Mara-Andreea Spataru}
\date{}

\begin{document}
	
	\maketitle
	
	A team of researchers from KAIST (Korea Advanced Institute of Science and Technology) led by a professor from the Department of Bio and Brain Engineering has made a revolutionary discovery in the field of oncology and molecular biology. 
	
	This team identified for the first time a molecular switch that was capable of inverting the transformation process of cancer cells, bringing them back to a similar state of normal cells. This discovery, published at the start of 2025, brings new perspectives in treating cancer. 
	
	So far, most of the treatments used to combat cancer concentrated on destroying the cancer cells through methods like chemotherapy, radiotherapy, or immunotherapy. The innovative approach proposed by the team is different, it doesn't aim at eliminating cancer cells, but transforming them back to cells with normal characteristics. This is actually a fundamental shift in oncology, from destroying to reprogramming.
	
	Researchers found that during the tumorigenesis (the process in which tumors are formed), there is a critical moment of transition during which normal cells start to transform into cancer cells. This critical transitional phenomenon can be compared to changing the state of water in steam when it reaches a temperature of 100°C. In this specific transitional point, the cells can be found in an unstable state, characterized by the coexistence of normal cells together with cancerous cells. The research team has succeeded to capture and analyze this transitional state using advanced methods of systemic biology.
	
	The research methodology involved the development of unique technology that automatically builds a computerized model of the genetic network that controls the critical transition in cancer development. This model was built using single-cell RNA sequencing data. Through simulation and computational analysis of this model, the researches have identified systematically molecular switches that can reverse the process of cancerization. 
	
	In order to validate their theoretical discoveries, the team has applied this technology on colon cancer cells. The molecular and cellular experiments have confirmed that cancerous cells can indeed get back their 'normal cells' characteristics when the respective molecular switches are activated accordingly. These results were obtained in collaboration with a research team from Seoul National University, which provided organoids (tissues cultured in vitro) from colon cancer patients. 
	
	The coordinating professor emphasizes the importance of this discovery and stated that this study has showed in detail what changes take place inside the cells during the process in which cancer develops, which so far was considered a mystery. 
	
	The implications of this particular discovery are vast and promising for the future of oncology. First of all, understanding the molecular mechanisms underlying the transition from normal to cancer cells provides new insight into cancer biology. Second of all, by identifying these molecular switches that can reverse this process opens up the way to developing new innovative therapies, targeted, therapies that have the potential to have fewer side effects compared to conventional treatments. 
	
	It is anticipated that this technology will be applied in the future for developing different therapies to reverse different types of cancers. Unlike traditional treatments that destroy cells and can harm healthy tissue, therapy based on reversing cancer cells offer a gentler and potentially more effective approach. The discovery of the molecular switch by KAIST team represents a major leap forward in understanding and treating cancer. By ``reprogramming'' in such way the cancer cells instead of destroying them, this research opens up the prospect of more effective and less invasive treatments, offering hope to millions of patients.
	
\vspace{2cm}
\begin{center}
	\textbf{References}
\end{center}

\begin{thebibliography}{9}
	
	\bibitem{source1} 
	News Medical Life Sciences. (2025, February 6). 
	\textit{KAIST team discovers molecular switch to reverse cancer cells}. 
	Retrieved from https://www.news-medical.net/news/20250206/KAIST-team-discovers-molecular-switch-to-reverse-cancer-cells.aspx
	
	\bibitem{source2}
	KAIST. (2025, February). 
	\textit{Professor Kwang-Hyun Cho's Research Team Discovers a Molecular Switch that Can Reverse Cancer}. 
	Retrieved from https://news.kaist.ac.kr/newsen/html/news/?mode=V\&mng\_no=42710
	
\end{thebibliography}

\end{document}